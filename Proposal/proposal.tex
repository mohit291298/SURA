\documentclass[11pt]{article}
\usepackage[utf8]{inputenc}
\usepackage{graphicx}
\usepackage{titlepic}
\usepackage{caption}
\usepackage{subcaption}
\usepackage[a4paper, total={6in, 8in}]{geometry}
\usepackage{hyperref}

% \documentclass{beamer}
\usepackage{amsmath}

\newcommand{\namesigdate}[2][5cm]{%
  \begin{tabular}{@{}p{#1}@{}}
    #2 \\[0.4\normalbaselineskip] \hrule \\[0pt]
    {\small } \\[2\normalbaselineskip] 
  \end{tabular}
}

\title{\vspace*{\fill} \textbf{Audio Description and Content Moderation App}
	  \\ {\large \textbf{Summer Undergraduate Research Award}}
	  \\  \vspace{3mm} \includegraphics[width=5cm]{logo.jpg}}

\author{
	\textbf{Ayush Patel}\\ 
	2016CS10396\\
	Computer Science\\
	CGPA: 9.305 \\
	Mob: 9891052662\\
	cs1160396@iitd.ac.in
	\and
	\textbf{Mohit Gupta}\\ 
	2016CS50433\\
	Computer Science\\
	CGPA: 9.579\\
	Mob: 9466479674\\
	cs5160433@iitd.ac.in
}
\date{\textbf{Supervisor:-} \\ \textbf{Aaditeshwar Seth} \\ Professor \\ Department of CSE \\ aseth@cse.iitd.ac.in\\ IIT Delhi\\
\vspace*{\fill}}




\begin{document}
	\maketitle

\begin{center}
\noindent\rule{3.2cm}{0.4pt} 
\end{center}

\begin{flushright}
\noindent\rule{3.2cm}{0.4pt} 
\\ \textbf{Prof. S. Arun Kumar}
\\ Head of Department
\\ Department of CSE
\\ sak@cse.iitd.ernet.in
\end{flushright}


	\newpage

	\section{Introduction}
	Citizen sourced journalism is an important medium of expression
    for people, especially in rural India, who are often kept out of discussions regarding
    issues faced by them. With increasingly improved technology and the penetration
    of mobile phones, giving people lot of computation power in their hands, the medium of the contribution of stories by the users is no longer limited to just text-based. People can also
    now contribute audios, videos, and photos to support their stories. But the underlying
    fact remains that whatever the degree of participation, and whatever the medium, there
    is a strong need of screening the contributed items, identifying the topics of discussion and grouping based on the same before they can be published. This is necessary to maintain the quality of the news of being reported, the decisions being carried out and take into account the sensitivities of various parties.
    \newline
    
    \begin{figure}
        \centering
        \includegraphics[width=15cm]{intro.png}
        \caption{Who knows your Media \textit{(cited from: newslaundry.com)}}
        \label{fig:media}
    \end{figure}
    
    
    Our app is built around the Mobile-Based Community Platforms. Take for example a platform which uses the common “missed call” concept where users place a call to a phone number and the server cuts the call and calls them back, thus making the system free of cost for the users. The IVR presents options to record voice messages in Hindi and English they want to share. A wide variety of topics can be featured on this app, including hyperlocal news, job openings, agriculture advisory, social issues such as early marriage and domestic violence, health Q&A, governance and accountability, folk songs and poems, and local and national level advertisements. The bulk of the content on the platform is user-generated with recordings contributed by the IVR itself and subsequently moderated and
    curated by the content team for publication on the IVR.\newline
    
    
    As the number of stories contributed by people rises, the need for moderation can turn
    into a bottleneck for scaling such platforms. It is important to ensure that the items
    worth publishing are accessible by other people soon after they are recorded. Our app
    aims to solve this problem by easing the task of the content moderation. In this scheme, moderation is done at two levels: first by community representatives with the help of the smart-phone app and second, by a central team of dedicated moderators. Different responsibilities are split across the two levels, as explained later.\newline
    
    
    According to surveys, in most of these current platforms, for more than 70\% of the rejected messages, the main reason for rejection is poor audio quality. Other messages get rejected because the report is not articulate enough, or it is incomplete, and only 1.5\% messages are rejected because the content is objectionable or incorrect. Our app also aims at addressing these problems. The MV content moderators are presented with this app which takes in all the recorded audios and categorizes each audio based on its quality by comparing it with a minimum threshold. This app then tries to enhance the poor-quality audios, and the ones who can’t be processed further are discarded. The app also has an abusive filter which follows advanced editorial policies to ensure that abusive words and statements are removed from content.\newline
    
    
    Currently, all that the community reporter sees is a long list of audios coming from the IVR. We also aim to add certain features to our app by which the stories contributed by the community reporter are better organized and hence aid him to follow-up on the issue much more effectively. Data and information in such an organized form can then also be shown to various concerned authorities, hence making the voices of the people to be heard where they need to be and improve chances
    of a resulting impact.\newline
    
    
    The app associates tags based on the broad topic/theme the audio belong to and thus organizes the data based on these tags. These tags help grouping similar kind of data and thus, a particular moderator specialized in that topic can go for the second level of screening of the same.
    

	\section{Objectives}
	Design, build and validate an app to moderate and curate the voice messages recorded on IVR (Interactive Voice Response) systems incorporated in Mobile-Based Community Platforms.

    The main objectives of the app is to to decentralize two decision points:
     \begin{itemize}
				\item
				        Selection of content
				\item
				        Ranking of content
			\end{itemize}
The app focuses on the "editorial problem" of selection of content to make sure that it is not abusive, it is not politically motivated, etc, and on how to rank the content.
\bigskip

The app will also provide indicators to the volunteers of the content is doing, i.e. after it was published, are people listening to it or skipping it. If a lot of people are skipping it then the volunteers should look more deeply into it, or the moderators can be consulted to investigate, and take corrective steps.

\bigskip

The app aims to satisfy the need of automatic content moderation in various Community Platforms by adding features to enhance audio quality, convert to text, abusive content filters, associates tags based on the broad topic/theme( and subtopics having to rank) the audio belongs to and thus, to organize audios collected via the app into issues and enable better tracking of the status of each.

\pagebreak
\section{Approach to the project}
    The basic purpose of the app is to provide the users with the moderated content by decentralizing the moderation and providing tools to community representatives (volunteers) to address the two decision point i.e. \textbf{``Selection of Content``} and \textbf{``Ranking of content``}\\
    \bigskip
    When an audio recorder by someone using IVR, it goes though audio enhancement process.\\
    As the audio is enhanced, all the moderators and community reporters get a notification about it and they can mark it as accepted or rejected. The moderator's response will help us to rank the content using a ranking mechanism. Then after publishing the audio, we also keep a track on it's popularity among users so as to make our system efficient in providing the users with the content that they want to listen.

        \begin{itemize}
			\item Audio Enhancement
			\begin{enumerate}
				\item
					We will use digital sampling to reconstruct the audio as much as possible.
				\item
					We sample at least twice as fast as the highest frequency we want to record so that we can use Nyquist theorem perfectly reconstruct the original sound wave from the spaced-out samples.				
			\end{enumerate}
			\item Abuse filtering
			\begin{enumerate}
				\item
					The user is asked to choose one of the provided languages so that it helps us to use a pre-trained neural network to convert audio to text.
                \item
					We will be using APIs like Watson speech to text or Google speech aip for converting the audio to text.
				\item
					Depending on the abusive words in the language, we segregate the audios into abusive tolerable or abusive intolerable.
			\end{enumerate}
			\item Hash-tag Assignment
			\begin{enumerate}
				\item
					We use "one vs all" multi-class classifier neural network for the purpose of hash-tag assignment and topic assignment to the audio.
                \item
					This will help user to search for the topics that they want to listen.
			\end{enumerate}
			\item Popularity Index of Audio among Users
			\begin{enumerate}
				\item
					We use an algorithm to provide indicators to the volunteers about how the content is doing, ie. after it was published, are people listening to it or skipping it.
                \item
					This will help volunteers to understand the taste of user and provide them with the contents that they want to listen.
			\end{enumerate}
			\item Further Possibilities
			\begin{enumerate}
				\item
				    We will be training on more languages to accommodate linguistic diversity prominently in rural areas
				\item
				    Apart from the app for content moderators, we may also build an app for users so that they can get news on demand as per their preferences.
			\end{enumerate}
		\end{itemize}


	\section{Budget and duration}	
		\subsection{Budget}
			No budget is required for this project.			
					
		\subsection{Duration}
			We aim to complete this project by the end of the summer break i.e. the end of July, 2017. 





	\begin{thebibliography}{1}
	
    \bibitem{}
    Design Lessons from Creating a Mobile Based Community Media Platform in Rural India
    \textit{Aparna
Moitra *, Vishnupriya Das **, Gram Vaani team **, Archna Kumar *, Aaditeshwar Seth **}

    \bibitem{}
    Reality Reporting and Moderation Apps for Community Reporters in Rural Areas
    \textit{Mridu Atray, Aaditeshwar Seth}
    
    \bibitem{}
    Elo rating system
    \textit{\href{https://en.wikipedia.org/wiki/Elo_rating_system}{https://en.wikipedia.org/wiki/Elo_rating_system}}


	\end{thebibliography}
\end{document}
