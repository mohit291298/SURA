\documentclass[11pt]{article}
\usepackage[utf8]{inputenc}
\usepackage{graphicx}
\usepackage{titlepic}
\usepackage{caption}
\usepackage{subcaption}
\usepackage[a4paper, total={6in, 8in}]{geometry}
\usepackage{hyperref}

% \documentclass{beamer}
\usepackage{amsmath}

\newcommand{\namesigdate}[2][5cm]{%
  \begin{tabular}{@{}p{#1}@{}}
    #2 \\[0.4\normalbaselineskip] \hrule \\[0pt]
    {\small } \\[2\normalbaselineskip] 
  \end{tabular}
}

\title{\vspace*{\fill} \textbf{Audio Description and Content Moderation App}
	  \\ {\large \textbf{Summer Undergraduate Research Award}}
	  \\  \vspace{3mm} \includegraphics[width=5cm]{logo.jpg}}

\author{
	\textbf{Ayush Patel}\\ 
	2016CS10396\\
	Computer Science\\
	CGPA: 9.305 \\
	Mob: 9891052662\\
	cs1160396@iitd.ac.in
	\and
	\textbf{Mohit Gupta}\\ 
	2016CS50433\\
	Computer Science\\
	CGPA: 9.579\\
	Mob: 9466479674\\
	cs5160433@iitd.ac.in
}
\date{\textbf{Supervisor:-} \\ \textbf{Aaditeshwar Seth} \\ Professor \\ Department of CSE \\ aseth@cse.iitd.ac.in\\ IIT Delhi\\
\vspace*{\fill}}




\begin{document}
	\maketitle

\begin{center}
\noindent\rule{3.2cm}{0.4pt} 
\end{center}

\begin{flushright}
\noindent\rule{3.2cm}{0.4pt} 
\\ \textbf{Prof. S. Arun Kumar}
\\ Head of Department
\\ Department of CSE
\\ sak@cse.iitd.ernet.in
\end{flushright}


	\newpage

	\section{Introduction}
	History proves that right from the starting of the time, Media had played an important role in influencing various sort of social, economical and political phenomenons in societies. It has led to union and division of society into various groups based on their opinions on different topics. It has led to many revolutionary movements through the masses. But some important question which arises are "Are all these news true?", "Can media even influence us to act opposite to our opinion?". Everyday we come accross various fake news in various unmoderated social media like Facebook and Twitter which leads to formation of various echo chambers. These tend to make false image of people and groups among the society. But moderated social media which tries to ensure that the information is credible and that it is presented with diverse viewpoints being covered, is hard to scale. This project aims to decentralize the moderation process so that it is easier to ensure the correctness and unbiasedness of the news been delivered. 
    \newline
    
    \begin{figure}
        \centering
        \includegraphics[width=15cm]{intro.png}
        \caption{Who knows your Media \textit{(cited from: newslaundry.com)}}
        \label{fig:media}
    \end{figure}
    
    
    This model consists of a community based news platform the news is collected through different user groups who record their audio over IVR. Before this audio is published, it is made sure that it is free of various issues viz-a-viz it is not abusive, it is not politically motivated etc. This working is made efficient using two levels of decentralization of moderation namely - community representatives (volunteers) and content moderators.

Community representatives forms the first level of decentralization which is closest to the user group. They listen to the audios recorded over IVR and then pass the relevenat content to the moderator. Then the content moderators, who form the second layer of decentralization, check the content for the political and social biasedness which is procedure for the selection of content, and then they rank the content based on it's importance and relevance. Then the overall ranking of content is determined by a ranking algoritm.

It is also necessary to ensure that the various levels of moderation are working efficiently. Our ground truth for the correctness of the decision taken by volunteers is based on the decision taken by the moderators for the same content. Based on this, the volunteers are provided scores based on a scoring algorithm. These scores play an important role in determining the incentives of the volunteers. So, to get better incentives, the volunteers will try to increase their score and in turn work properly. This will increase the efficiency of our system.

A system can be said to be successfull only if it suffices for the need of the users. To measure the popularity of the content among the users, a mathematical measure called Popularity Index is determined based on the lifetime of a content viz-a-viz whether people are listening to the content or skipping it etc. Then while publishing, the content which has high popularity index is preferrd over the one with less popularity index. This system provides moderated unbiased content to the user based on their preferrence.
    

	\section{Objectives}
	Design, build and validate an app to moderate and curate the voice messages recorded on IVR (Interactive Voice Response) systems incorporated in Mobile-Based Community Platforms.

    The main objectives of the app is to to decentralize two decision points:
     \begin{itemize}
				\item
				        Selection of content
				\item
				        Ranking of content
			\end{itemize}
The app focuses on the "editorial problem" of selection of content to make sure that it is not abusive, it is not politically motivated, etc, and on how to rank the content.
\bigskip

The app will also provide indicators to the volunteers of the content is doing, i.e. after it was published, are people listening to it or skipping it. If a lot of people are skipping it then the volunteers should look more deeply into it, or the moderators can be consulted to investigate, and take corrective steps.

\bigskip

The app aims to satisfy the need of automatic content moderation in various Community Platforms by adding features to enhance audio quality, convert to text, abusive content filters, associates tags based on the broad topic/theme( and subtopics having to rank) the audio belongs to and thus, to organize audios collected via the app into issues and enable better tracking of the status of each.

\pagebreak
\section{Approach to the project}
    The overall approach is to provide the content recored by the user group over IVR to the community representatives (voluenteers) who pass the relevant audios to the content moderators. These content moderators do the selection and ranking of the content based on various factors such as biasedness of the content, it's political impact etc. The overall ranking of the content is determined with the help on a ranking algorithm. Then the content finally been published is available to the user on demand. Then based on the lifetime of the content, a mathematical measure called Popularity Index of content is determined through which the community representatives get to know the need of user group and then refine the content accordingly. To ensure the efficient working of the volunteers, they are given scores based on their work which in turn determines the incentive that they get for their work.

        \begin{itemize}
			\item Collection of Content


			We use an IVR (Interactive Voice Response) to get audio content from the user group. Interactive Voice Response (IVR) is an automated telephony system that interacts with callers, gathers information and routes calls to the appropriate recipient. The user gives a missed call to a number and the the computer calls back the user asap and then the user can speak about the content and the system records it.


			\item Moderation of Content


			The process of moderation is decentralized in order to make it's working more efficient. It consists of two lavels of decentralizartion - Community Representatives (Volunteers) and Content Moderators.


			\begin{enumerate}
				\item Moderation by Community Representatives


			As soon as the audio file is recorded over IVR, all the community representatives (volunteers) receive a notification about the content in their phone and any of them can listen to it. The volunteers check if the audio file is empty or is it of very poor quality or if it's to abusive or politically motivated. Based on these, they pass only the relevant contents to the next level of decentralization i.e. content moderators.


				\item Moderation by Content Moderators


			Moderator looks at the contents passed on by community representatives. They basically perform following two tasks namely - selection of content and ranking of content. They do the selection based on various factors such as biasedness of the content, it's political impact etc. Accordingly they reject the contents that are not suitable to be published. Then among the contents which are to be published, they rank them based on their understanding. The overall ranking of the content is determined with the help on a ranking algorithm which takes into account the ranks given to a content by each content moderator. 
			\end{enumerate}


			\item Publishing Content


			The content which passed the moderation test are published in order of their ranking. The user can listen to content on demand, according to their preference. The user response i.e. wheteher a user listens to or skip a content is recorded and this information is used in the next step (Popularity Index).


			\item Content Popularity Index


			Based on the information of lifetime of the content viz-a-viz whether people are listening to the content or skipping it etc, a mathematical measure of popularity of content called Popularity Index is calculated. Based on the measure of polularity indices of contents of a particular topics, volunteers determine wheteher user likes to listen about the topic or not. This helps to get the information about the demands of user group and suggests volunteers to prefer contents on certain topics over the others.


			\item Scoring System for Volunteers


			It might happen that volunteers don't work efficiently and just randomly passes the content to content moderators. In order to check this, sometimes randomly some contents are sent both to volunteers and content moderators. If the decision of a volunteer about a content is same as the decision taken by the moderator, his/her score increases or else the score decreases. This is done with the help of a scoring algoritm. This score inturn decides the incentive that the volunteer gets for his/her work. So, in order to increase his/her incentive, the volunteer will try to increase his/her score and hence will work more efficiently.


		\end{itemize}


	\section{Budget and duration}	
		\subsection{Budget}
			No budget is required for this project.			
					
		\subsection{Duration}
			We aim to complete this project by the end of the summer break i.e. the end of July, 2017. 





	\begin{thebibliography}{1}
	
    \bibitem{}
    Design Lessons from Creating a Mobile Based Community Media Platform in Rural India
    \textit{Aparna
Moitra *, Vishnupriya Das **, Gram Vaani team **, Archna Kumar *, Aaditeshwar Seth **}

    \bibitem{}
    Reality Reporting and Moderation Apps for Community Reporters in Rural Areas
    \textit{Mridu Atray, Aaditeshwar Seth}
    
    \bibitem{}
    Elo rating system
    \textit{\href{https://en.wikipedia.org/wiki/Elo_rating_system}{https://en.wikipedia.org/wiki/Elo_rating_system}}


	\end{thebibliography}
\end{document}
